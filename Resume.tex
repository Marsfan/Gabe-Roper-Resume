% !TEX TS-program = xelatex
% !TEX encoding = UTF-8 Unicode
% -*- coding: UTF-8; -*-
% vim: set fenc=utf-8

%%%%%%%%%%%%%%%%%%%%%%%%%%%%%%%%%%%%%%%%%%%%%%%%%%%%%%%%%%%%%%%%%
%% CV.tex
%% <https://github.com/zachscrivena/simple-resume-cv>
%% This is free and unencumbered software released into the
%% public domain; see <http://unlicense.org> for details.
%%%%%%%%%%%%%%%%%%%%%%%%%%%%%%%%%%%%%%%%%%%%%%%%%%%%%%%%%%%%%%%%%

% See "README.md" for instructions on compiling this document.

\documentclass[letterpaper,ddMMMyyyy,nonstopmode]{simpleresumecv}
\usepackage[document]{ragged2e}
\usepackage{fix-cm}

%%%%%%%%%%%%%%%%%%%%%%%%%%%%%%%%%%%%%%%%%%%%%%%%%%%%%%%%%%%%%%%%%%%%%%%%%%
%%	Uncomment to display a grid on the document to help with alinment	%%
%%%%%%%%%%%%%%%%%%%%%%%%%%%%%%%%%%%%%%%%%%%%%%%%%%%%%%%%%%%%%%%%%%%%%%%%%%
%\usepackage{background}
%
%\newlength\mylen
%\setlength\mylen{\dimexpr\paperwidth/80\relax}
%
%\SetBgScale{1}
%\SetBgAngle{0}
%\SetBgColor{blue!30}
%\SetBgContents{\tikz{\draw[step=\mylen] (-.5\paperwidth,-.5\paperheight) grid (.5\paperwidth,.5\paperheight);}}



% Class options:
% a4paper, letterpaper, nonstopmode, draftmode
% MMMyyyy, ddMMMyyyy, MMMMyyyy, ddMMMMyyyy, yyyyMMdd, yyyyMM, yyyy

%%%%%%%%%%%%%%%%%%%%%%%%%%%%%%%%%%%%%%%%%%%%%%%%%%%%%%%%%%%%%%%%%
%% PREAMBLE.
%%%%%%%%%%%%%%%%%%%%%%%%%%%%%%%%%%%%%%%%%%%%%%%%%%%%%%%%%%%%%%%%%

% CV Info (to be customized).
\newcommand{\CVAuthor}{Gabriel Roper}
\newcommand{\CVTitle}{Gabriel Roper's Resume}
\newcommand{\CVNote}{CV compiled on {\today}}
\newcommand{\LinkedInPage}{https://www.linkedin.com/in/gabriel-roper/}
\newcommand{\GithubPage}{https://www.github.com/Marsfan}
\newcommand{\emailAddress}{roperg@my.erau.edu}
\newcommand{\streetAddress}{}
\newcommand{\phoneNumber}{469-264-4980}
\newcommand{\descriptionWidth}{33em}
\newcommand{\tableWidth}{23.5em}



% PDF settings and properties.
\hypersetup{
	pdftitle={\CVTitle},
	pdfauthor={\CVAuthor},
	pdfsubject={\LinkedInPage},
	pdfcreator={XeLaTeX},
	pdfproducer={},
	pdfkeywords={},
	unicode=true,
	bookmarks=true,
	bookmarksopen=true,
	pdfstartview=FitH,
	pdfpagelayout=OneColumn,
	pdfpagemode=UseOutlines,
	hidelinks,
	breaklinks}

% Shorthand.
% \newcommand{\Code}[1]{\mbox{\textbf{#1}}}
% \newcommand{\CodeCommand}[1]{\mbox{\textbf{\textbackslash{#1}}}}

\newenvironment{Description}
{%
	\SmallGap
	\par
	\begin{Detail}
		\Item
		\begin{minipage}{\descriptionWidth}
}
{\par
\end{minipage}
\end{Detail}
}


%%%%%%%%%%%%%%%%%%%%%%%%%%%%%%%%%%%%%%%%%%%%%%%%%%%%%%%%%%%%%%%%%
%% ACTUAL DOCUMENT.
%%%%%%%%%%%%%%%%%%%%%%%%%%%%%%%%%%%%%%%%%%%%%%%%%%%%%%%%%%%%%%%%%

\begin{document}
	%%%%%%%%%%%%%%%
	% TITLE BLOCK %
	%%%%%%%%%%%%%%%
	\Title{\CVAuthor}

	\begin{SubTitle}
		{\streetAddress}
		\par
		\href{mailto:\emailAddress}{\emailAddress}
		\BulletSymbol
		\,\,\phoneNumber\,
		\BulletSymbol
		\,\,\href{\LinkedInPage}{\url{\LinkedInPage}}
		\BulletSymbol
		\,\,\href{\GithubPage}{\url{\GithubPage}}
	\end{SubTitle}

	\begin{Body}

		%%%%%%%%%%%%%%%
		%% EDUCATION %%
		%%%%%%%%%%%%%%%

		\Section
		{Education}
		{Education}
		{PDF:Education}

		\Entry
		\href{https://prescott.erau.edu}{\textbf{Embry-Riddle Aeronautical University}}
		\hfill Prescott, Arizona

		Bachelor of Science in Electrical Engineering
		\hfill Expected May 2021

		Concentration: Robotics
		\hfill GPA: 3.605

		%%%%%%%%%%%%%%%%%%%%%%%%%
		%% Leadership Activities %%
		%%%%%%%%%%%%%%%%%%%%%%%%%

		\Section
		{Leadership/ Activities}
		{Leadership/Activities}
		{PDF:Leadership/Activities}

		\Entry


		\Entry
		\textbf{EagleSat}
		\hfill
		\yearRange{2017}{Present}

		\begin{Description}
				Member of the Electrical Power System team, developing the power subsystems for a student-designed and built CubeSat that will analyze the effect of cosmic radiation on flash memory and look for cosmic rays.
		\end{Description}
		\Gap

		\Entry
		\textbf{Eagle Aero Sport}
		\hfill
		\yearRange{2017}{Present}

		\begin{Description}
				Member of electrical engineering team that is mounting a number of sensors to a student-built Van's RV-12 Light Sport Aircraft for analyzing the stress and aerodynamic effects on the aircraft.
		\end{Description}
		\Gap

		%\Entry
		%\textbf{Coppell High School FIRST Robotics Team}
		%\hfill\yearRange{2013}{2017}

		%\begin{Description}
		%		Student-Built Android based robots that compete in First Tech Challenge.
		%		\BulletItem{Programming Leader, Fall 2015-Sprint 2017}
		%		\BulletItem{Co-Captain, Fall 2016-Spring 2017}
		%		\BulletItem{Fundraising Leader, Sprint 2014-Spring 2017}
        %\end{Description}
        %\Gap

        \Entry
        \textbf{Eagle Amateur Radio Club}
        \hfill\yearRange{2017}{Present}
        \begin{Description}
            Student-run organization that help promot amateur radio at the university and provides access to amateur radio hardware for students to use.
            \BulletItem{Vice President, Fall 2018-Spring 2019}

        \end{Description}

		\Gap

		\Entry
		\textbf{Boy Scout Troop 845, Boy Scouts of America, Circle 10 Council}
		\hfill\yearRange{2013}{2017}

		\begin{Description}
				Youth Organization that promotes leadership and responsibility.
				\BulletItem{Eagle Scout}
				\BulletItem{Patrol Leader, Purple Platypi, 2013-2014}
				\BulletItem{Interim Senior Patrol Leader, July 2014 Summer Camp}
		\end{Description}


		%%%%%%%%%%%%%%%%%%%%%
		%% Research		   %%
		%%%%%%%%%%%%%%%%%%%%%
		\Section
		{Research}
		{Research}
		{PDF:Research}

		\Entry
		\textbf{Pulsed Plasma Thruster}
		\hfill
		\yearRange{2019}{Present}

		\begin{Description}
			Project currently developing a pulsed plasma thruster for cubesat spacecraft.
		\end{Description}

		%%%%%%%%%%%%%%%%%%%%%
		%% Work Experience %%
		%%%%%%%%%%%%%%%%%%%%%

		\Section
		{Work Experience}
		{Work Experience}
		{PDF:Work Experience}
		\Entry\textbf{L3Harris ACSS}\hfill\ymRange{2019}{05}{2019}{08}
			\positionSummary{Electrical Engineering Intern}

			\begin{Description}
				Developed Software to automate production and testing of commercial aircraft collision avoidance systems.
			\end{Description}
		\Gap


		\Entry\textbf{National Aeronautics and Space Administration}\hfill\ymRange{2018}{06}{2018}{08}\\
		\positionSummary{Human-Spacesuit Integration Intern -- Anthropometrics and Biomechanics Facility}

		\begin{Description}
			Developed software for analyzing human-spacesuit interaction to improve the ergonomics of spacesuits.
		\end{Description}
		\Gap

		\Entry\textbf{Usedave.com}\hfill\ymRange{2015}{06}{2015}{08}
		\positionSummary{Freelance 3D Modeling}

		\begin{Description}
				Created 3D models using Blender and Sketchup for corporate internal marketing comic book.

		\end{Description}



		%%%%%%%%%%%%%%%%%%%%%%%%%%%
		%%       Skills          %%
		%%%%%%%%%%%%%%%%%%%%%%%%%%%
		\Gap
		\Section{Skills}{Skills}{PDF: Skills}
		% The hspace is needed here to shift this over correctly.
		\def\arraystretch{0}%  1 is the default, change whatever you need
		\hspace*{-0.8em}
		\begin{tabular}[t]{p{6em} p{\tableWidth}}
			\textit{Engineering Software} &
			\begin{Detail}
				SOLIDWORKS, KiCAD, LabVIEW, MakerBot Desktop, CATIA, Eagle, PyCharm, IAR Embedded Workbench, Git
			\end{Detail}\\
			\textit{Office Software} &
			\begin{Detail}
				Microsoft Word, Microsoft Excel, Microsoft PowerPoint, Microsoft Publisher, LibreOffice Suite
			\end{Detail}\\
			\textit{Other Software}&
			\begin{Detail}
				Blender Animation\newline~\newline
			\end{Detail}\\
			\textit{Programming Languages} &
			\begin{Detail}
				Arduino, Python, MATLAB, C, Basic VHDL\newline~\newline
			\end{Detail}\\
			\textit{Technical}&
			\begin{Detail}
				Woodworking, Soldering, PCB Design, 3D Printing
			\end{Detail}
		\end{tabular}


		\Section{Licenses and Certifications}{Licenses and Certifications}{PDF: Licenses and Certifications}

		\Entry FCC Amateur Radio Technician
		\Entry CATIA Associate - Part Design
		\Entry CITI Program Responsible Conduct in Research
		\Entry SOLIDWORKS Mechanical Design Associate


		%%%%%%%%%%%%
		%% AWARDS %%
		%%%%%%%%%%%%

		\Section
		{Awards}
		{Awards}
		{PDF:Awards}
		\Entry NASA High School Aerospace Scholar
		\Entry National Merit Commended Student
	\end{Body}

\end{document}
